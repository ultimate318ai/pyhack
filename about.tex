\documentclass[a4paper,10pt]{article}
%\documentclass[a4paper,10pt]{scrartcl}

\usepackage[utf8]{inputenc}
\usepackage{pifont}
\usepackage{graphicx}
\usepackage{pgf,tikz}
\usepackage{amsmath}
\usepackage{amssymb}
\usepackage{textcomp}

\title{Pyhack : A new strategic game }
\author{Nathan Daunois}
\date{20/12/19}

\pdfinfo{%
  /Title    (pyhack)
  /Author   (Nathan DAUNOIS)
  /Creator  (Nathan DAUNOIS)
  /Producer ()
  /Subject  ()
  /Keywords ()
}

\begin{document}
\maketitle



 
 \begin{enumerate}
\item Introduction  
\item Principe du jeu
\item Les commandes
\item Monstres et objets
\item Calculs score
 \end{enumerate}

\newpage 

\section{Le jeu}
\subsection{Introduction}

Le jeu Pyhack est inspiré des graphismes du jeu Nethack , avec cependant des modifications au niveau du fonctionnement du jeu.

Le joueur doit survivre le plus longtemps possible , il commence dans une salle puis doit passer dans l'étage superieur .
le nombre de salle augmente petit à petit ainsi que la difficulté des monstres et à certains niveaux , des Monstres spéciaux 
apparaissent ainsi que des Boss !! 

Les 10 meilleurs scores sont répertoriés dans la partie ``Bests players'' du menus ! 
A vous de jouer et .... Bonne chance ! :)

\subsection{Les commandes }

\subparagraph{Dans le menus du jeu :}

 Dans le menus , il faut sse deplacer avec $2$ et $8$ pour aller respectivement vers le haut et le bas.
 On valide avec la touche enter .
 
\subparagraph{Dans le jeu :}
Dans le jeu , il y a plus de commandes :
\begin{itemize}
 \item Se deplacer (en ligne droite)  : fleches directionnelles
 \item Se deplacer (en diagonale) :7 (haut gauche) , 9 (haut droit),1 (bas gauche) , 3 (bas droit)
 \item enlever/mettre les stats : e
 \item attaquer (contact du monstre) : c
 \item attaquer (distance) : d
 \item quitter : q
\end{itemize}

\subparagraph{Dans le ``champ vision attaque distance'' :}
\begin{itemize}
 \item se deplacer : 8 : haut , 4 : gauche , 6 : droite , 2 : bas 
 \item attaquer la cible : d
 \item quitter : q
\end{itemize}


\newpage

\section{Monstres et objets}

\subsection{Monstres}

\subparagraph{Les types de monstres}
sont diverse , on a les sbires (lettres de a a z) en orange , puis en rouge , puis en noir , difficulté croissante 
celons la lettre et aussi la couleur ! \\
Certains monstres sont spéciaux , comme par exemple le $\frac{1}{2}$ qui enlève la moitié de la vie au joueur donc attention !

\subparagraph{Les boss apparaissent tous les 5 niveaux}
ils sont plus fort que les autres monstres mais rapportent beaucoup d'xp mais attaquent au contact et 
a distance !

\subsection{objets}

Les objets ont des fonctionnalités diverse :

\begin{itemize}
 \item le $\textcolor{green}{+}$ : ajoute de la vie
 \item le $\textcolor{green}{=}$ : remet la vie au max
 \item le $\textcolor{red}{?}$ : ajoute (ou enleve ! ) un nombre aléatoire de pv 
 \item le $\textcolor{green}{!}$ : ajoute un nombre aléatoire de pv
 \item le $\textcolor{black}{\%}$ : augmente l'attaque de 1
 \item le $\textcolor{blue}{[]}$ : augmente l'armure de 2
 \item le $\textcolor{blue}{()}$ : idem mais de 1
 \item le $\textdollar$ : fait apparaitre les escaliers  dans un salle aléatoirement!
 \item le \# : les fameux escaliers : fait passer au niveau suivant (impossible de revenir en arrière ! ) 
\end{itemize}


Les montres enlevent une quatité de vie au joueur en fonction de son armure (NB : les monstres spèciaux ne tiennent pas 
compter de l'armure ! )

\subsection{Calcul du score}

Le score est calculé de la manière suivante : 

$$ \varphi = xp_{joueur} + niveau_{jeu}^{2} + niveau_{joueur}^{3} $$




\end{document}
